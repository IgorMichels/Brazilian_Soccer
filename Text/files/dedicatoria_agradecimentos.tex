\begin{dedicatoria}
    \vspace*{\fill}
    \hfill
    \begin{minipage}{.6\textwidth}
    Dedico essa dissertação à minha avó Ângela (\textit{in memoriam}), que sempre me incentivou a estudar e foi fundamental para que eu estivesse aqui.
    \end{minipage}
\end{dedicatoria}
 
\begin{agradecimentos}
    Gostaria de agradecer a todas as pessoas que contribuíram comigo durante toda essa jornada, em especial:
    
    Meus pais, Patrício e Dilma, bem como minha irmã Natália e meu avô João, por todo incentivo aos estudos e por sempre apoiar minhas escolhas, estando sempre ao meu lado, mesmo eu estando distante de casa.
    
    A minha namorada, Fernanda, que me apoia sempre, até mesmo nas ideias mais malucas, além de dar sua mais sincera opinião acerca das versões prévias deste trabalho.
    
    Aos amigos que reencontrei e as novas amizades que fiz durante esse período, os quais auxiliaram tanto com suporte emocional, por estar distante de minha família, quanto com discussões sobre os tópicos visto em sala.
    
    Ao Centro de Desenvolvimento da Matemática e Ciências da FGV, que acreditou em meu potencial e me auxiliou, tanto emocionalmente quanto financeiramente, durante esses quatro anos.
    
    Ao Conselho Nacional de Desenvolvimento Científico e Tecnológico (CNPq) pela concessão de uma bolsa de estudos durante meu projeto de iniciação científica, o qual culminou nesse trabalho.
    
    Por fim gostaria de agradecer a todos os professores que, de alguma forma, contribuíram para minha formação. Em especial gostaria de agradecer a meu orientador, Moacyr Silva, por concordar em me auxiliar na condução desse trabalho, sugerindo algumas melhorias.
\end{agradecimentos}

\begin{epigrafe}
\vspace*{\fill}

\begin{flushright}
    \hspace{7.5cm}
    \textit{
        ``A vida é como um jogo de futebol, cada lance pode definir sua trajetória.''} \\
        \textit{Michael Johnathon}
\end{flushright}
\end{epigrafe}