\chapter{Introdução}
\label{intro}

O futebol é o esporte mais popular do mundo. Inventado na Inglaterra durante o século XIX e trazido ao Brasil por meio do paulista Charles William Miller, em 1894 \cite{wikifutebol}, esse esporte virou paixão nacional. Tal paixão acaba unindo pessoas de diversas classes sociais, raças, credos e regiões em virtude de uma partida ou de um título de um clube.

Além da parte social do esporte, o futebol também tem muito dinheiro envolvido. Um estudo recente da CBF em parceria com consultoria ``EY'' mostrou que o futebol não é mais apenas um lazer, mas também um negócio que representa 0,72\% do PIB brasileiro \cite{oeconomista}. Esse montante se dá não apenas pelo valor arrecadado em dias de jogos (ingressos e toda a logística necessária para a partida), mas também com o aumento do consumo em restaurantes e bares, nos quais muitos torcedores acabam por assistir as partidas e discutir sobre o jogo com outros torcedores. Conversas como ``que time é melhor?'', ``quem vai ser o campeão do campeonato X?'' ou ``o jogador A é melhor que o jogador B'' surgem quase que naturalmente, muitas vezes à base do achismo.

Dentro do contexto esportivo, outro ramo que vem crescendo muito nos últimos anos é o mercado de apostas. Por ser o esporte mais popular do mundo, o mesmo corresponde a maior fatia desse mercado, que deve chegar a cerca de 180 bilhões de dólares em 2030 \cite{gvr}. Para se ter uma ideia de quão grande esse mercado já está, a Copa do Mundo de 2018, realizada na Rússia, rendeu a Fifa pouco menos de 6 bilhões de euros. Por outro lado, o volume global de apostas vinculadas a competição foi cerca de 22 vezes maior \cite{gaz}.

Dessa forma, o estudo de modelos matemáticos voltados ao esporte está se tornando mais relevante, tanto para levar um embasamento estatístico para conversas quanto para previsão esportiva, seja aplicada as casas de apostas ou junto ao apostador. Nesse trabalho buscamos modelar o futebol brasileiro a nível individual, isso é, buscaremos responder perguntas como ``quais são os melhores jogadores que atuam no Brasil?'' e ``o jogador A é melhor que o jogador B?''.

Esse trabalho se organiza por meio da introdução, seguida da revisão de literatura e da descrição do processo de raspagem dos dados, o qual possibilitou toda a modelagem proposta. Após o processo de raspagem seguimos para a modelagem, a qual se divide em premissas, tratamento dos dados e nas duas abordagens exploradas, a frequentista e a bayesiana.