\chapter{Conclusão}

A estatística é uma das principais ferramentas para análise de dados e inferência. No presente trabalho analisamos sua aplicação no contexto do futebol brasileiro, uma vez que, com tais análises, pode-se ter um melhor embasamento durante discussões acerca o futebol, além de, é claro, no contexto das casas de apostas, onde cada informação a mais pode ser relevante, tanto para o apostador quanto para a banca.

Para tanto, o trabalho se iniciou com a raspagem e tratamento dos dados disponíveis nas súmulas da CBF, processo esse que acabou por ser automatizado com o intuito de obter os dados conforme os mesmos fossem disponibilizados.

Após a raspagem dos dados iniciou-se o processo de modelagem, no qual abordamos o problema tanto no paradigma frequentista quanto no bayesiano. Infelizmente a abordagem frequentista não funcionou conforme o esperado, uma vez que a otimização da função de verossimilhança se mostrou extremamente custosa. Desse modo tal abordagem foi deixada de lado.

Por outro lado, a abordagem bayesiana foi implementada, com três modelos propostos, e executada sem grandes problemas. O primeiro modelo modela os gols da partida simplesmente por meio dos jogadores que estavam em campo no momento do gol, com cada jogador tendo uma proficiência de ataque e outra de defesa. Já o segundo modelo discrimina o mando de campo, isso é, atribuí diferentes parâmetros para os jogadores conforme o local da partida, o que fez com que cada atleta possuísse quatro parâmetros. Por fim, o último modelo buscou tratar a questão do mando de campo de modo a não inflar muito o número de parâmetros, assim tal modelo interpretou o clube mandante como um novo jogador, o qual atua a favor do mandante da partida, como uma partida 12 contra 11.

Ainda assim, em virtude do alto número de parâmetros a serem estimados, a complexidade do programa ficou alta. Para contornar o problema computacional o código foi rodado nos servidores do GitHub o que possibilitou, além do maior poder computacional de tais servidores, a paralelização de alguns dos processos, fazendo com que cada servidor executasse algumas cadeias de Markov em paralelo.

Essa abordagem acabou gerando resultados interessantes, como o ranking dos melhores jogadores e a comparação de dois atacantes que foram referência no Brasil em 2022: Pedro e Germán Cano, onde, além de comparar, vimos a diferença entre o desempenho de ambos jogando em casa ou fora de casa. Por fim, também foi analisado o comportamento do fator mando de campo por meio do ranking das equipes que mais se valem de tal fator.

\chapter{Trabalhos Futuros}

Por se tratar de um trabalho envolvendo extração de dados, o número de possibilidades de expansão é grade. Entre elas, pode-se citar a análise de redes voltadas aos clubes e jogadores, como a análise de quais clubes mais realizam transferência de jogadores entre si, ou a clusterização de jogadores, por meio de uma rede onde os jogadores são nós e as arestas são ponderadas de acordo com o número de partidas (ou minutos) que cada jogador passou em campo tendo outro jogador como companheiro de time, por exemplo.

Além dos projetos envolvendo redes, outras possibilidades são a aplicação de diferentes modelos nos dados, seja para modelar os jogadores ou para predição, valendo ressaltar que o trabalho de predição necessita de alguma heurística para definir as escalações de cada time para previsões de longo prazo. No caso do modelo frequentista essa modelagem pode ser realizada por meio da correção ou reimplementação do modelo aqui apresentado, talvez com alguma outra clusterização ou uso de outro otimizador. Já em relação a abordagem bayesiana, a mesma pode ser variada por meio da modelagem dos gols ou da priori utilizada. Ao variar a priori utilizada, um trabalho interessante pode ser a tentativa de precificar os jogadores, através da utilização do valor de mercado como priori.

Ainda no âmbito de previsões, uma possibilidade é a extensão do processo de raspagem para possibilitar a coleta das informações sobre os técnicos, possibilitando utilizar essa informação na modelagem. Tal ideia se dá pela grande mudança de postura observada recentemente em algumas equipes, como no Flamengo, onde a mudança de técnico, no ano de 2022, foi determinante na forma de jogar da equipe, saindo de uma equipe próxima do rebaixamento para a conquista da Copa do Brasil e da Copa Libertadores.
