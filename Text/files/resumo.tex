\setlength{\absparsep}{18pt} 
\begin{resumo}[Resumo]

O futebol é uma paixão nacional. O Brasil, apesar de não ser o berço desse esporte, é conhecido como ``país do Futebol'', sendo natural que muitas conversas surjam com essa natureza. Em geral tais conversas abrangem tópicos como ``qual o melhor time'', ``quem irá ganhar uma determinada partida ou competição'' ou então comparando dois jogadores, avaliando qual é o melhor ou o preferido para um contratação do seu clube de coração. Nesse sentido, o estudo desse esporte, a nível de modelagem, pode trazer um embasamento matemático à conversa, deixando o ``achismo'' de lado.

No presente trabalho analisou-se o desempenho dos jogadores das Séries A e B do campeonato brasileiro de futebol, organizado pela CBF (Confederação Brasileira de Futebol), sob o ponto de vista da Inferência Bayesiana. Para tanto, realizou-se a raspagem dos dados das súmulas de tais competições.

No contexto da modelagem três modelos foram propostos, um utilizando apenas os dados das equipes que foram a campo, bem como o resultado dessas partidas, enquanto os outros dois também se valiam da informação do mando de campo, com o intuito de modelar o diferente comportamento, dentro e fora de casa. Os principais resultados se dão pela estimação de distribuições de parâmetros que quantificam o desempenho dos jogadores, possibilitando, dessa forma, comparar e ranquear os mesmos, além de ranquear os clubes com maior influência dentro de seus domínios.

Palavras-chave: Modelagem matemática. Futebol. Inferência Bayesiana.
\end{resumo}

\begin{resumo}[Abstract]
    \begin{otherlanguage*}{english}
        Soccer is a national passion. Brazil, in spite of not being the birthplace of this sport, is known as the ``country of Soccer'', and it is natural that many conversations arise with this theme. In general, such conversations cover topics such as ``which team is the best'', ``who will win a certain match or competition'' or comparing two players, evaluating which is the best or the preferred one for signing your favorite club. In this sense, the study of this sport, at the modeling level, can bring a mathematical basis to the conversation, leaving ``guessing'' aside.

        In this work, we analyzed the performance of players from Series A and B of the Brazilian soccer championship, organized by the CBF (Confederação Brasileira de Futebol), from the point of view of Bayesian inference. For that, the data of the summaries of such competitions was scraped.
        
        In the context of the modeling, three models were proposed, one using only the line-up's, as well as the result of these matches, while the other two also used information from the field command, in order to model the different behavior, inside and outside the house. The main results are given by the estimation of distributions of parameters that quantify the performance of the players, making it possible, in this way, to compare and rank them, in addition to ranking the clubs with greater influence within their domains.
    \end{otherlanguage*}

Keywords: Mathematical modeling. Soccer. Bayesian inference.
\end{resumo}